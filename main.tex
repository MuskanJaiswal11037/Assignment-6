

\usepackage[utf8]{inputenc}
\documentclass{beamer}
\usetheme{CambridgeUS}
\usepackage{listings}
\usepackage{blkarray}
\usepackage{listings}
\usepackage{subcaption}
\usepackage{url}
\usepackage{tikz}
\usepackage{tkz-euclide} % loads  TikZ and tkz-base
%\usetkzobj{all}
\usetikzlibrary{calc,math}
\usepackage{float}
\renewcommand{\vec}[1]{\mathbf{#1}}
\usepackage[export]{adjustbox}
\usepackage[utf8]{inputenc}
\usepackage{amsmath}
\usepackage{amsfonts}
\usepackage{tikz}
\usepackage{hyperref}
\usepackage{bm}
\usetikzlibrary{automata, positioning}
\providecommand{\pr}[1]{\ensuremath{\Pr\left(#1\right)}}
\providecommand{\mbf}{\mathbf}
\providecommand{\qfunc}[1]{\ensuremath{Q\left(#1\right)}}
\providecommand{\sbrak}[1]{\ensuremath{{}\left[#1\right]}}
\providecommand{\lsbrak}[1]{\ensuremath{{}\left[#1\right.}}
\providecommand{\rsbrak}[1]{\ensuremath{{}\left.#1\right]}}
\providecommand{\brak}[1]{\ensuremath{\left(#1\right)}}
\providecommand{\lbrak}[1]{\ensuremath{\left(#1\right.}}
\providecommand{\rbrak}[1]{\ensuremath{\left.#1\right)}}
\providecommand{\cbrak}[1]{\ensuremath{\left\{#1\right\}}}
\providecommand{\lcbrak}[1]{\ensuremath{\left\{#1\right.}}
\providecommand{\rcbrak}[1]{\ensuremath{\left.#1\right\}}}
\providecommand{\abs}[1]{\vert#1\vert}

\newcounter{saveenumi}
\newcommand{\seti}{\setcounter{saveenumi}{\value{enumi}}}
\newcommand{\conti}{\setcounter{enumi}{\value{saveenumi}}}
\usepackage{amsmath}
\setbeamertemplate{caption}[numbered]{}


\title{\typedef{ASSIGNMENT 6}}       
\author{MUSKAN JAISWAL -cs21btech11037}
\date{May 2022}
\logo{\large \Latex{}}
\begin{document}

\begin{frame}{Outline}
  \tableofcontents
\end{frame}
\section{Abstract}
	\begin{frame}{Abstract}
		\begin{itemize}
			\item 	This document contains the explanation of question  7.18  of Papoulis Pillai Probability book of chapter sequence of random variables.
		\end{itemize}
	\end{frame}
	
\maketitle

\section{QUESTION:}
\begin{frame}{}
\begin{block}{}
Show that if $a_0 + a_1x_1+ a_2x_2$ is the non-homogeneous linear MS(Mean Square) estimate of s in terms of 
$x_1$and $X_2$,then \\
$\hat{E}\{s-n_s|x_1-n_1,x_2-n_2\}=\alpha_1(x_1-n_1)+\alpha_2(x_2-n_2) \\$
\end{block}
\end{frame}
\section{ANSWER:}
\begin{frame}
\begin{align*}
\hat{E}(\frac{s}{x_1,x_2})=&a_0+a_1_x_1+a_2x_2\\
\hat{E}(\frac{s}{x_1-n_1,x_2-n_2})=&a_0+a_1(x_1-n_1)+a_2(x_2-n_2)\\
\hat{E}(\frac{s-n_s}{x_1-n_1,x_2-n_2})=&\hat{E}(\frac{s}{x_1-n_1})-\hat{E}({n_s}{x_1-n_1,x_2-n_2})\\
=&a_0+a_1(x_1-n_1)+a_2(x_2-n_2)-\hat{E}(a_0)-\hat{E}(x_1-n_1)\\
\tab \tab -\hat{E}(x_2-n_2)\\
=&a_0+a_1(x_1-n_1)+a_2(x_2-n_2)-a_0\\
=&a_1(x_1-n_1)+a_2(x_2-n_2)
\end{align*}
\end{frame}
\end{document}
